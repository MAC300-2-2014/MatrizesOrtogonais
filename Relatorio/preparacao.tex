\chapter{Método QR}

	\section{Decomposição da matriz A}
	Se A for uma matriz não-singular então é possível decompô-la em um produto de matrizes 
	$QR$, onde $Q$ é ortogonal e $R$ é uma matriz triangular superior de diagonal não-nula. Esta decomposição é única e sempre existe.
	
	Para obter essa decomposição deve-se encontrar uma sequência de $Q_i ^T$, $i~ =~ 1,...,m$ onde m é o número de colunas do sistema, sendo que cada uma dessas $Q_i^T$ transforma cada $i$-ésima coluna em zeros abaixo da $i$-ésima posição, formando assim a matriz triangular superior $R$.\cite{sawp} Tendo assim que
	
	\[Q_{n-1} ^TQ_{n-2} ^T \ldots Q_{1} ^T A = R\]
	
	Como cada $Q_i ^T$ é ortogonal, temos $A = QR$, onde \[Q = [Q_{n-1} ^T Q_{n-2} ^T \ldots Q_{1} ^T]^T \]
	\section{Cálculo do vetor x}
	Para resolver um sistema $Ax=b$ podemos decompor a matriz $A$ em termos de $Q$ e $R$ como \[ A = Q{R \brack 0}\] com a matriz $Q$ sendo quadrada de dimensão $n\times n$ e $R$ triangular-superior de dimensão~ $n\times m$.
	A equação toma a forma \[Q^Tx = b\] que pode ser escrito como \[Rx=Q^Tb\]
	Pode-se então separar a solução em dois passos:
	\begin{enumerate}
	\item Calcular $y = Q^Tb$
	\item Resolver o sistema $Rx = y$
	\end{enumerate}
	
	No desenvolvimento do algoritmo não se calculou explicitamente as matrizes $Q$ e $R$ para aumentar sua eficiência. A forma com que os valores necessários para o cálculo da solução são armazenados e utilizados está descrito no algoritmo para decomposição $QR$.
	
	\section{Reutilização do QR para qualquer b}
	Uma vez calculada a decomposição $QR$ de uma matriz $A$, basta realizar os passos anteriores e obter novas soluções variando apenas os dados de $b$.

	\begin{enumerate}
	\item Calcular $y = Q^Tb$
	\item Resolver o sistema $Rx = y$
	\end{enumerate}
	
	Dessa forma, pode-se reaproveitar todo o processamento das matrizes $QR$ e reduzir o tempo de resolução do algoritmo
	a apenas o cálculo de um produto matriz vetor e uma solução de matriz triangular superior.
	Caso não houvesse esse reaproveitamento, a cada novo vetor de dados $b$ para uma mesma matriz $A$ seriam necessários todos os cálculos para decomposição, o que levaria muito processamento desnecessário.
\chapter{Método dos Quadrados Mínimos}

	\section{Descrição}
	O Método dos Quadrados Mínimos é uma técnica de otimização matemática que procura encontrar o melhor ajuste para um 
	conjunto de dados tentando minimizar a soma dos quadrados das diferenças entre o valor estimado e os dados observados.
	Tais diferenças são chamadas resíduos.
	
	
	Consiste em um estimador que minimiza a soma dos quadrados dos resíduos, de forma a maximizar o grau de ajuste do modelo aos dados observados.\cite{wikipedia}
	
	\section{Ordem do algoritmo}
	Calculando a ordem do algoritmo, temos:
	
	Passo 1: ordem $nm$;

	Passo 2: ordem $[(nm)+(m-1)+(m-2)+\ldots +1] \approx nm + m^2$;
	
	Passo 3: ordem $nm$;
	
	Passo 4: ordem $n + (n-1) + \ldots + 1 \approx n^2$;
	
	Passo 5: ordem $3nm + 3(n-1)(m-1) + \ldots + 3 \approx nm^2$;
	
	No total: $\approx m^2+n^2+3nm+m^2 n$.



\chapter{Algoritmo para decomposição QR}


A decomposição da matriz $A \in R^{n\times m}$
 em $QR$, $Q \in R^{n\times n}$, $R \in R^{n\times m}$, $n > m$ segue os seguintes passos:
\newline


1) Determinar o elemento de maior valor $max$ da matriz $A$ e determinar $Â = \frac{1}{max}A$;

Loop ($i = 0 ... m$)

2) Calcular a norma das colunas da submatriz $Â_{(n-i)\times (m-i)}$ e armazenar no vetor sigma;

3) Permutar a coluna $i$ com a que possuir maior norma em sigma (busca nos índices $k = i \ldots m$). Caso a maior norma seja menor que epsilon parar;

4) Encontrar $u, u^T e \gamma $ para determinar $Q= I - \gamma u u^T$;

5) Calcular $QÂ_i$ e armazenar $u$.
\newline


O passo 1 tem por objetivo evitar overflow nos cálculos das normas, pois $\forall a_{ij} \le |1|$.

A iteração sobre o passo 2 e 3 'empurram’ possíveis pivôs nulos para a matriz, no caso de $A$ possuir posto incompleto. Para o recálculo das normas basta extrair os elementos de cada coluna da linha anterior a partir da iteração $i = 1 \ldots r$ ($r$ = posto de $A$).

No passo 5, é possível calculá-lo de forma econômica processando  $QA= A - u((\gamma  u^T)A)$

Como as linhas da coluna $i$ serão zeradas, nestas posições são armazenadas o vetor u.

