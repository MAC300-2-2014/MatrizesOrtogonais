% !TeX spellcheck = en_US
%% abtex2-modelo-relatorio-tecnico.tex, v-1.9.2 laurocesar
%% Copyright 2012-2014 by abnTeX2 group at http://abntex2.googlecode.com/ 
%%
%% This work may be distributed and/or modified under the
%% conditions of the LaTeX Project Public License, either version 1.3
%% of this license or (at your option) any later version.
%% The latest version of this license is in
%%   http://www.latex-project.org/lppl.txt
%% and version 1.3 or later is part of all distributions of LaTeX
%% version 2005/12/01 or later.
%%
%% This work has the LPPL maintenance status `maintained'.
%% 
%% The Current Maintainer of this work is the abnTeX2 team, led
%% by Lauro César Araujo. Further information are available on 
%% http://abntex2.googlecode.com/
%%
%% This work consists of the files abntex2-modelo-relatorio-tecnico.tex,
%% abntex2-modelo-include-comandos and abntex2-modelo-references.bib
%%

% ------------------------------------------------------------------------
% ------------------------------------------------------------------------
% abnTeX2: Modelo de Relatório Técnico/Acadêmico em conformidade com 
% ABNT NBR 10719:2011 Informação e documentação - Relatório técnico e/ou
% científico - Apresentação
% ------------------------------------------------------------------------ 
% ------------------------------------------------------------------------

\documentclass[
	% -- opções da classe memoir --
	12pt,				% tamanho da fonte
	openright,			% capítulos começam em pág ímpar (insere página vazia caso preciso)
	oneside,			% para impressão em verso e anverso. Oposto a oneside
	a4paper,			% tamanho do papel. 
	% -- opções da classe abntex2 --
	%chapter=TITLE,		% títulos de capítulos convertidos em letras maiúsculas
	%section=TITLE,		% títulos de seções convertidos em letras maiúsculas
	%subsection=TITLE,	% títulos de subseções convertidos em letras maiúsculas
	%subsubsection=TITLE,% títulos de subsubseções convertidos em letras maiúsculas
	% -- opções do pacote babel --
	english,			% idioma adicional para hifenização
	french,				% idioma adicional para hifenização
	spanish,			% idioma adicional para hifenização
	brazil,				% o último idioma é o principal do documento
	]{abntex2}


% ---
% PACOTES
% ---

% ---
% Pacotes fundamentais 
% ---
\usepackage{lmodern}			% Usa a fonte Latin Modern
\usepackage[T1]{fontenc}		% Selecao de codigos de fonte.
\usepackage[utf8]{inputenc}		% Codificacao do documento (conversão automática dos acentos)
\usepackage{indentfirst}		% Indenta o primeiro parágrafo de cada seção.
\usepackage{color}				% Controle das cores
\usepackage{graphicx}			% Inclusão de gráficos
\usepackage{microtype} 			% para melhorias de justificação
\usepackage{amssymb}
\usepackage[boxruled, linesnumbered, portuguese]{algorithm2e}
\usepackage{alltt}
% ---

% ---
% Pacotes adicionais, usados no anexo do modelo de folha de identificação
% ---
\usepackage{multicol}
\usepackage{multirow}
% ---
	
% ---
% Pacotes adicionais, usados apenas no âmbito do Modelo Canônico do abnteX2
% ---
\usepackage{lipsum}				% para geração de dummy text
% ---

% ---
% Pacotes de citações
% ---
\usepackage[brazilian,hyperpageref]{backref}	 % Paginas com as citações na bibl
\usepackage[alf]{abntex2cite}	% Citações padrão ABNT

% --- 
% CONFIGURAÇÕES DE PACOTES
% --- 

% ---
% Configurações do pacote backref
% Usado sem a opção hyperpageref de backref
\renewcommand{\backrefpagesname}{Citado na(s) página(s):~}
% Texto padrão antes do número das páginas
\renewcommand{\backref}{}
% Define os textos da citação
\renewcommand*{\backrefalt}[4]{
	\ifcase #1 %
		Nenhuma citação no texto.%
	\or
		Citado na página #2.%
	\else
		Citado #1 vezes nas páginas #2.%
	\fi}%
% ---

% ---
% Informações de dados para CAPA e FOLHA DE ROSTO
% ---
\titulo{Matrizes Ortogonais e o \\Problema de Quadrados Mínimos}
\autor{Florence Alyssa Sakuma Shibata \and Shayenne da Luz Moura}
\local{São Paulo}
\data{2014}
\instituicao{%
  Universidade de São Paulo -- USP
  \par
  Instituto de Matemática e Estatística
  \par
  Bacharelado em Ciência da Computação}
\tipotrabalho{Relatório técnico}
% O preambulo deve conter o tipo do trabalho, o objetivo, 
% o nome da instituição e a área de concentração 
\preambulo{Relatório dos resultados dos testes do segundo trabalho menor.}
% ---

% ---
% Configurações de aparência do PDF final

% alterando o aspecto da cor azul
\definecolor{blue}{RGB}{41,5,195}

% informações do PDF
\makeatletter
\hypersetup{
     	%pagebackref=true,
		pdftitle={\@title}, 
		pdfauthor={\@author},
    	pdfsubject={\imprimirpreambulo},
	    pdfcreator={LaTeX with abnTeX2},
		pdfkeywords={abnt}{latex}{abntex}{abntex2}{relatório técnico}, 
		colorlinks=true,       		% false: boxed links; true: colored links
    	linkcolor=blue,          	% color of internal links
    	citecolor=blue,        		% color of links to bibliography
    	filecolor=magenta,      		% color of file links
		urlcolor=blue,
		bookmarksdepth=4
}
\makeatother
% --- 

% --- 
% Espaçamentos entre linhas e parágrafos 
% --- 

% O tamanho do parágrafo é dado por:
\setlength{\parindent}{1.3cm}

% Controle do espaçamento entre um parágrafo e outro:
\setlength{\parskip}{0.2cm}  % tente também \onelineskip

% ---
%% compila o indice
%% ---
\makeindex
% ---

% ----
% Início do documento
% ----
\begin{document}

% Retira espaço extra obsoleto entre as frases.
\frenchspacing 

% ----------------------------------------------------------
% ELEMENTOS PRÉ-TEXTUAIS
% ----------------------------------------------------------
\pretextual

% ---
% Capa
% ---
\imprimircapa
% ---

% ---
% Folha de rosto
% (o * indica que haverá a ficha bibliográfica)
% ---
%\imprimirfolhaderosto*
% ---


% ---
% RESUMO
% ---

% resumo na língua vernácula (obrigatório)
\setlength{\absparsep}{18pt} % ajusta o espaçamento dos parágrafos do resumo
\begin{resumo}
Este relatório trata do desenvolvimento de um algoritmo eficiente e eficaz
para resolução de sistemas lineares com posto completo e incompleto. Para 
tanto, utilizou-se o método QR com permutação de colunas. A análise dos
cálculos realizados pelo algoritmo foi implementada utilizando o problema
dos quadrados mínimos. Por fim, todos os resultados foram condizentes com 
o esperado.

 \noindent
 \textbf{Palavras-chaves}: QR. problema quadrados mínimos. resolução de sistemas lineares.
\end{resumo}
% ---

%% ---
%% inserir lista de ilustrações
%% ---
%\pdfbookmark[0]{\listfigurename}{lof}
%\listoffigures*
%\cleardoublepage
%% ---
%
%% ---
%% inserir lista de tabelas
%% ---
%\pdfbookmark[0]{\listtablename}{lot}
%\listoftables*
%\cleardoublepage
%% ---
%
%% ---
%% inserir lista de abreviaturas e siglas
%% ---
%\begin{siglas}
%  \item[ABNT] Associação Brasileira de Normas Técnicas
%  \item[abnTeX] ABsurdas Normas para TeX
%\end{siglas}
%% ---
%
%% ---
%% inserir lista de símbolos
%% ---
%\begin{simbolos}
%  \item[$ \Gamma $] Letra grega Gama
%  \item[$ \Lambda $] Lambda
%  \item[$ \zeta $] Letra grega minúscula zeta
%  \item[$ \in $] Pertence
%\end{simbolos}
%% ---
%
%% ---
% inserir o sumario
% ---
\pdfbookmark[0]{\contentsname}{toc}
\tableofcontents*
\cleardoublepage
%% ---
%
%
%% ----------------------------------------------------------
%% ELEMENTOS TEXTUAIS
%% ----------------------------------------------------------
%\textual

% ----------------------------------------------------------
% Introdução
% ----------------------------------------------------------
\chapter*[Introdução]{Introdução}
\addcontentsline{toc}{chapter}{Introdução}

Em investigações científicas é comum buscar relações entre dados pontuais. Normalmente tem-se 
muitos pontos coletados em experimento e há sempre uma razão teórica para acreditar 
que esses pontos são aproximados por uma função.

Como é praticamente impossível que uma função atravesse exatamente todos os pontos de dados
é interessante encontrar uma função que tenha a uma distância aceitável, ou seja, que
possua o menor desvio médio possível.



Encontrar essa função é, na realidade, encontrar a solução de um sistema linear sobredeterminado,
isto é, um sistema com mais equações que incógnitas, $Ax = b$, onde 
$A \in  \mathbb{R}^{n \times m}, n \ge m \ e \ b \in \mathbb{R}^n$. Sendo $A$ a matriz que define o sistema,
$n$ o número de equações, $m$ o número de incógnitas e $b$ os dados do problema. 

Para analisar quão boa é essa aproximação utilizou-se o método dos quadrados mínimos. Para tal, deve-se encontrar não a função que atravessa os pontos mas a que minimiza as distâncias entre a solução calculada e os valores observados. 

Neste caso, o problema consiste em encontrar $x \in \mathbb{R}^n$ para o qual $\|r\|_2$ é minimizada, onde $r =  b-Ax$ é o vetor de 
resíduos. 

Para resolver esse problema, utilizou-se o conceito de matriz ortogonal, que é uma matriz quadrada cuja
inversa coincide com sua transposta, por esta possuir número de condição igual a 1, isto é, é bem condicionada.
Matrizes ortogonais nos permitem realizar operações com o sistema $Ax=b$ sem perder a precisão de sua solução.







% ----------------------------------------------------------
% PARTE - preparação da pesquisa
% ----------------------------------------------------------
\part{Teoria do algoritmo}

% ----------------------------------------------------------
% Capitulo com exemplos de comandos inseridos de arquivo externo 
% ----------------------------------------------------------

\chapter{Método QR}

	\section{Decomposição da matriz A}
	Se A for uma matriz não-singular então é possível decompô-la em um produto de matrizes 
	$QR$, onde $Q$ é ortogonal e $R$ é uma matriz triangular superior de diagonal não-nula. Esta decomposição é única e sempre existe.
	
	Para obter essa decomposição deve-se encontrar uma sequência de $Q_i ^T$, $i~ =~ 1,...,m$ onde m é o número de colunas do sistema, sendo que cada uma dessas $Q_i^T$ transforma cada $i$-ésima coluna em zeros abaixo da $i$-ésima posição, formando assim a matriz triangular superior $R$. Tendo assim que
	
	\[Q_{n-1} ^TQ_{n-2} ^T \ldots Q_{1} ^T A = R\]
	
	Como cada $Q_i ^T$ é ortogonal, temos $A = QR$, onde \[Q = [Q_{n-1} ^T Q_{n-2} ^T \ldots Q_{1} ^T]^T \]\textbf{http://www.sawp.com.br/blog/?p=689}%\cite{sawp}
	\section{Cálculo o vetor x}
	Para resolver um sistema $Ax=b$ podemos decompor a matriz $A$ em termos de $Q$ e $R$ como \[ A = Q{R \brack 0}\] com a matriz $Q$ sendo quadrada de dimensão $n\times n$ e $R$ triangular-superior de dimensão~ $n\times m$.
	A equação toma a forma \[QTx = b\] que pode ser escrito como \[Rx=Q^Tb\]
	Pode-se então separar a solução em dois passos:
	\begin{enumerate}
	\item Calcular $y = Q^Tb$
	\item Resolver o sistema $Rx = y$
	\end{enumerate}
	
	No desenvolvimento do algoritmo não calculou-se explicitamente as matrizes $Q$ e $R$ para aumentar sua eficiência. A forma com que os valores necessários para o cálculo da solução são armazenados e utilizados está descrito no algoritmo para decomposição $QR$.
	
	\section{Reutilização do QR para qualquer b}
	Uma vez calculada a decomposição $QR$ de uma matriz $A$, basta realizar os passos anteriores e obter novas soluções variando apenas os dados de $b$.

	\begin{enumerate}
	\item Calcular $y = Q^Tb$
	\item Resolver o sistema $Rx = y$
	\end{enumerate}
	
	Dessa forma, pode-se reaproveitar todo o processamento das matrizes $QR$ e reduzir o tempo de resolução do algoritmo
	a apenas o cálculo de um produto matriz vetor e uma solução de matriz triangular superior.
	Caso não houvesse esse reaproveitamento, a cada novo vetor de dados $b$ para uma mesma matriz $A$ seriam necessários todos os cálculos para decomposição, o que levaria muito processamento desnecessário.
\chapter{Método dos Quadrados Mínimos}

	\section{Descrição}
	O Método dos Quadrados Mínimos é uma técnica de otimização matemática que procura encontrar o melhor ajuste para um 
	conjunto de dados tentando minimizar a soma dos quadrados das diferenças entre o valor estimado e os dados observados.
	Tais diferenças são chamadas resíduos.
	
	
	Consiste em um estimador que minimiza a soma dos quadrados dos resíduos, de forma a maximizar o grau de ajuste do modelo aos dados observados.\cite{wikipedia}



\chapter{Algoritmo para decomposição QR}
Por questão da otimização de código e eficiência de cálculos, a matriz $Q$ é representada em forma $Q = (I - \gamma uu^T)$. A prova de que a matriz $Q$ pode ser escrita nesta forma está desenvolvida em \cite{watkins}.

A ordem do algoritmo de decomposição QR com posto incompleto consiste em:
\begin{enumerate}
	\item Determinar o elemento de maior valor absoluto $max$ da matriz $A_{nm}$
	\item Multiplicar a matriz A por $\frac{1}{max}$ 
	
	\textbf{Inicia o loop.}
	  \item Calcular/Recalcular a norma das colunas de $A_{nm}$
	  \item Permutar a coluna atual com a coluna de maior norma
	  \item Encontrar a $Q$ e fazer $QA$
	  
	\textbf{Finaliza o loop.}
\end{enumerate}

Os passos 1 e 2 do algoritmo tem por finalidade evitar overflow no cálculo
das normas, uma vez que todos os elementos da matriz, após o passo 2, 
serão <= $\|1\|$ (menor igual a módulo de 1).

A iteração sobre os passos 3 e 4 'empurram' possíveis pivôs nulos para a direita 
da matriz no caso da matriz $A$ possuir posto incompleto. 
As normas ao quadrado de cada coluna são armazenadas em um vetor separadamente. Para
o seu recálculo da iteração~$i$~($i$~>~2), apenas subtrai-se os elementos de cada coluna da 
linha anterior a partir de $i$. Após a permutação da coluna, caso a norma seja equivalente
a zero (menor que um certo epsilon), significa que as submatrizes de $A$ a serem decompostas 
são nulas, uma vez que a maior delas vale zero.

No passo 5, para encontrar a matriz $Q = (I - \gamma uu^T)$, basta apenas encontrar
$\gamma$, $u$ e $\sigma$ (norma de $u$). Não é necessário calcular explicitamente Q. 
Para o produto $QA$ a cada iteração, temos: $QA$ = $A - u  ((\gamma  u^T)  A)$.
Na iteração $i$, como as linhas da coluna $i$ serao zeradas, nestas posições sao armazenadas
o vetor $u$.

\section{Solução dos Mínimos Quadrados com QR}
Dada uma matriz $Ax = b$; $A^{n \times m}$;$ b^{n \times 1}$; $n > m$,
é desejado minimizar  $||r|| = || Ax - b ||$.  Utilizando o fato
de que a matriz Q e ortogonal, tem-se  $|| Ax - b || = || Q^T QRx - Q^Tb || = || Rx - Q^Tb ||$.

Assim, para posto completo minimizar $||r||$ implica em:
$||r|| =  ||c - Rx|| + ||d||$
[formula 3.3.5 p. 229 do livro Watkins]


para posto incompleto:
[p. 233 do livro Watkins, antes do teorema 3.3.12]


onde $x_2$ foi escolhido arbitrariamente como [0 ... 1 ... 1]
                                     indice 1 ... r ... m
				     $r$ = posto de $A$

\part{Testes}

\chapter{Resolução de sistemas}
Aqui estão descritos os testes do algoritmo de acordo com a matriz $A$ dada.

\section{Posto completo}
A entrada do algoritmo recebeu uma matriz A com posto completo para realizar
o cálculo da solução e seu resíduo em quadrados mínimos.

A saída do programa foi a seguinte:
\section{Posto incompleto}

A entrada do algoritmo recebeu uma matriz A com posto imcompleto para realizar
o cálculo da solução e seu resíduo em quadrados mínimos.

A saída do programa foi a seguinte:

\chapter{Entrada consistente}
A inconsistência da entrada pode modificar e até mesmo anular os resultados
obtidos através do algoritmos. Por isso, é necessário que o mesmo possua 
métodos para impedir que dados inconsistentes sejam calculados, tornando
a solução mais confiável e o algoritmo eficaz.

\chapter{Efeitos da perturbação}
Os dados raramente são exatos ao serem representados no computador, até mesmo na 
coleta destes podem haver erros de aproximação e medida que podem interferir no
resultado esperado do sistema.
Agora será descrito o que ocorre ao perturbar os valores das entradas e 
comparar as soluções anteriores e o quanto isso influencia no cálculo das 
soluções encontradas.


\section{Nos valores de A}

\section{Nos valores de b}

\section{Nos valores de A e b}

\section{No resíduo dos quadrados mínimos}

\chapter{Diminuir ou aumentar o grau do polinômio}

\chapter{Base que estabiliza a matriz}


% ----------------------------------------------------------
% Parte de revisãod e literatura
% ----------------------------------------------------------
\part{Resultados}

% ---
% Capitulo de revisão de literatura
% ---
\chapter{Resolução de sistema com posto completo}


\chapter{Resolução de sistema com posto incompleto}


\chapter{Problema dos quadrados mínimos}


\chapter{Eficiência e tolerancia a falhas}
% ---
% Finaliza a parte no bookmark do PDF
% para que se inicie o bookmark na raiz
% e adiciona espaço de parte no Sumário
% ---
\phantompart

% ---
% Conclusão
% ---
\chapter*[Conclusão]{Conclusão}
\addcontentsline{toc}{chapter}{Conclusão}

\lipsum[31-33]

% ----------------------------------------------------------
% ELEMENTOS PÓS-TEXTUAIS
% ----------------------------------------------------------
\postextual

% ----------------------------------------------------------
% Referências bibliográficas
% ----------------------------------------------------------
\bibliography{references}

% ----------------------------------------------------------
% Glossário
% ----------------------------------------------------------
%
% Consulte o manual da classe abntex2 para orientações sobre o glossário.
%
%\glossary

% ----------------------------------------------------------
% Apêndices
% ----------------------------------------------------------

% ---
% Inicia os apêndices
% ---
%\begin{apendicesenv}
%
%% Imprime uma página indicando o início dos apêndices
%\partapendices
%
%% ----------------------------------------------------------
%\chapter{Quisque libero justo}
%% ----------------------------------------------------------
%
%\lipsum[50]
%
%% ----------------------------------------------------------
%\chapter{Nullam elementum urna vel imperdiet sodales elit ipsum pharetra ligula
%ac pretium ante justo a nulla curabitur tristique arcu eu metus}
%% ----------------------------------------------------------
%\lipsum[55-57]
%
% \end{apendicesenv}
%% ---
%
%
% ----------------------------------------------------------
% Anexos
% ----------------------------------------------------------

% ---
% Inicia os anexos
% ---
\begin{anexosenv}

% Imprime uma página indicando o início dos anexos
\partanexos

% ---
\chapter{Morbi ultrices rutrum lorem.}
% ---
\lipsum[30]

% ---
\chapter{Cras non urna sed feugiat cum sociis natoque penatibus et magnis dis
parturient montes nascetur ridiculus mus}
% ---

\lipsum[31]

% ---
\chapter{Fusce facilisis lacinia dui}
% ---

\lipsum[32]

\end{anexosenv}
%

\end{document}
