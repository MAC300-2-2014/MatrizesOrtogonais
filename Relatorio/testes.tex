\chapter{Testes de eficiência do algoritmo}
Aqui estão descritos os testes do algoritmo de acordo com a matriz $A$ dada.

\section{Posto completo}
A entrada do algoritmo recebeu uma matriz A com posto completo para realizar
o cálculo da solução e seu resíduo em quadrados mínimos.

A saída do programa foi a seguinte:
\section{Posto incompleto}

A entrada do algoritmo recebeu uma matriz A com posto imcompleto para realizar
o cálculo da solução e seu resíduo em quadrados mínimos.

A saída do programa foi a seguinte:

\section{Entrada consistente}
A inconsistência da entrada pode modificar e até mesmo anular os resultados
obtidos através do algoritmos. Por isso, é necessário que o mesmo possua 
métodos para impedir que dados inconsistentes sejam calculados, tornando
a solução mais confiável e o algoritmo eficaz.


\chapter{Efeitos da perturbação em diferentes funções}
Os dados raramente são exatos ao serem representados no computador, até mesmo na 
coleta destes podem haver erros de aproximação e medida que podem interferir no
resultado esperado do sistema.
Agora será descrito o que ocorre ao perturbar os valores das entradas e 
comparar as soluções anteriores e o quanto isso influencia no cálculo das 
soluções encontradas.

\section{Diminuir ou aumentar o grau do polinômio}
Soluções com diferentes graus polinomiais podem aproximar de forma diferente
os dados do problema.
Os graus em que foram testados o algoritmo foram 2, 3, 4 e 5.
A partir destas foram feitas perturbações nos dados para comparar os resultados.

\section{Nos valores de b}
Os dados do vetor de entrada b foram perturbados em 15\% e foram mantidos os
valores de A.
A compararação dos resultados foram feitas de acordo com o grau do polinômio. 
\section{Nos valores de A e b}

Os dados do vetor de entrada de A e b foram perturbados em 15\%.
A compararação dos resultados foram feitas de acordo com o grau do polinômio.

\chapter{Base que estabiliza a matriz}

