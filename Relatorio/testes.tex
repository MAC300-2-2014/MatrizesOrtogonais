\chapter{Testes de eficiência do algoritmo}
Aqui estão descritos os testes do algoritmo de acordo com a matriz $A$ dada.

\section{Posto completo}
A entrada do algoritmo recebeu uma matriz $A$ com posto completo para realizar
o cálculo da solução e seu resíduo em quadrados mínimos e o vetor $b$.

Para a construcao do suposto polinômio de grau 5, foi construida a matriz $Ax = b$
onde:

\[
A =	\left(
\begin{array}{ccccc} 
1&	x_1^1&	x_1^2&	x_1^3&	x_1^4\\
1&	x_2^1&	x_2^2&	x_2^3&	x_2^4\\
&&\ldots\\
1&	x_k^1&	x_k^2&	x_k^3&	x_k^4\\
\end{array}
\right)
b =   
\left(
\begin{array}{c}
y_1\\
y_2\\
\ldots\\
y_{k-1}\\
y_k 
\end{array}    
\right)
\]

Comparando-o ao polinômio original e verificando se os coeficientes 
são razoavelmente próximos do original.


Para a construção do polinômio de grau 4 foi construida a matriz $Ax = b$ de forma
semelhante ao de grau 5:
\[A =	
\left(
\begin{array}{cccc}
1 &	x_1^1 &	x_1^2 &	x_1^3\\
1 &	x_2^1 &	x_2^2 &	x_2^3\\
&&\dots          \\
1 &	x_k^1 &	x_k^2 &	x_k^3\\
\end{array} 
\right)
b =  
\left(
\begin{array}{c}
y_1\\
y_2\\
\ldots\\
y_{k-1}\\
y_k\\ 
\end{array}   
\right)
\]
A mesma construção foi utilizada para polinômios de grau 3 e 2.


\section{Posto incompleto}

A entrada do algoritmo recebeu uma matriz A com posto imcompleto para realizar
o cálculo da solução e seu resíduo em quadrados mínimos.


\chapter{Efeitos da perturbação em diferentes funções}
Os dados raramente são exatos ao serem representados no computador, até mesmo na 
coleta destes podem haver erros de aproximação e medida que podem interferir no
resultado esperado do sistema.
Agora será descrito o que ocorre ao perturbar os valores das entradas e 
comparar as soluções anteriores e o quanto isso influencia no cálculo das 
soluções encontradas.

\section{Diminuir ou aumentar o grau do polinômio}
Soluções com diferentes graus polinomiais podem aproximar de forma diferente
os dados do problema.
Os graus em que foram testados o algoritmo foram 2, 3, 4 e 5.
A partir destas foram feitas perturbações nos dados para comparar os resultados.

\section{Nos valores de b}
Os dados do vetor de entrada b foram perturbados em 15\% e foram mantidos os
valores de A.
A compararação dos resultados foram feitas de acordo com o grau do polinômio. 
\section{Nos valores de A e b}

Os dados do vetor de entrada de A e b foram perturbados em 15\%.
A compararação dos resultados foram feitas de acordo com o grau do polinômio.



